\documentclass[12pt]{article}

%\usepackage[utf8]{inputenc}
%\usepackage{amsmath}
%\usepackage{graphicx}
\usepackage{hyperref}
\usepackage[document]{ragged2e}
\usepackage{listings}
\usepackage{xcolor} % for backgroundcolor
\definecolor{ForestGreen}{RGB}{34, 139, 34}
%\graphicspath{ {imgs/} }

\title{Recursive macros}
\date{}
\author{}
\begin{document}

\lstset {
    language=C++,
    backgroundcolor=\color{black!9},
    basicstyle=\footnotesize,
    keywordstyle=\color{blue}\ttfamily,
    stringstyle=\color{red}\ttfamily,
    commentstyle=\color{ForestGreen}\ttfamily,
    stepnumber=1,
}

\maketitle

\begin{FlushLeft}

\begin{tabular}{|l|l|}
\hline
 Document number: & p??? \\ \hline
 Date: & 2023-10-19  \\ \hline
 Project: & Programming language C++ \\ \hline
 Audience: & EWG \\ \hline
 Author: & Kelbon \\ \hline
 Reply-to: & Kelbon \href{mailto:kelbonage@gmail.com}{kelbonage@gmail.com} \\ \hline
\hline
\end{tabular}

\end{FlushLeft}

\newpage

\tableofcontents
\newpage

\section{Abstract}
We propose a special \lstinline{__THIS_MACRO__} identifier, which makes it possible to create recursive macros.

\section{Motivation}
 There are huge amount of code, which uses code generation, fragile and confusing macros
  or misuse templates for code generation
\begin{itemize}
  \item \href{https://github.com/nlohmann/json/blob/836b7beca4b62e2a99465edef44066b7401fd704/include/nlohmann/detail/macro_scope.hpp#L320}{nlohmann json[link]}
  \item \href{https://github.com/133794m3r/c-library/blob/master/p99/p99_generated.h}{P99[link]}
  \item \href{https://github.com/Hirrolot/metalang99/blob/695a9d8dbc3507447ba28d312da229730d20c6fc/include/metalang99/eval/rec.h#L12}{Metalang99[link]}
  \item \href{https://github.com/boostorg/preprocessor/blob/develop/include/boost/preprocessor/seq/fold_left.hpp}{Boost.Preprocessor[link]}
  \item \href{https://github.com/boostorg/pfr/blob/develop/include/boost/pfr/detail/core17_generated.hpp}{generated code in boost::pfr[link]}
  \item \href{https://github.com/kelbon/AnyAny/blob/4b056be2b6cbcfa1a407f7ee75279af414e390e4/include/anyany/noexport/data_parallel_vector_details.hpp#L62}{generated code in AnyAny[link]}
\end{itemize}

Such code is difficult to maintain, test, write and debug.
It's easy to make a mistake and compiler is often unable to produce a good error.

\medskip

But people need these opportunities.

\bigskip


Existing solutions do not allow programmers to avoid the problems described above,
and code generation adds its own: once a code generation script has been written,
it requires support in order to be able to change the generated code in the future.

\bigskip

Modern C++ provides opportunities to get rid of the preprocessor where it was never needed,
but there will always be places where the preprocessor is a best reflection of intentions.

\bigskip

Also C++20 adds \lstinline{__VA_OPT__}, which is ideal for recursive macros, but there are no such thing in C++!

\newpage

\section{Proposed solution}

It is proposed to add a special \lstinline{__THIS_MACRO__} token, which is behaves as macro name in macro definition,
except name of the macro containing the \lstinline{__THIS_MACRO__} token in the definition replaced in 
\href{https://eel.is/c++draft/cpp.rescan}{[cpp.rescan]} even if it is found during scan of the replacement list

\medskip

This makes possible to write such macros:

\begin{lstlisting}[language=C++]
#define fold_right(op, head, ...) \
( head __VA_OPT__(op __THIS_MACRO__(op, __VA_ARGS__)) )
/* expands to (1 + (2 + (3))) */
int i = fold_right(+, 1, 2, 3);
\end{lstlisting}

Calculate count of tokens
\begin{lstlisting}[language=C++]
#define TOKCOUNT_IMPL(head, ...) \
1 __VA_OPT__(+ __THIS_MACRO__(__VA_ARGS__))
/* works for zero args too */
#define tokcount(...) \
(0 __VA_OPT__(+ TOKCOUNT_IMPL(__VA_ARGS__)) )

/* expands to (0 + 1 + 1 + 1 ) */
tokcount(1, A, "gdfdg")

\end{lstlisting}

Replace for boost::pfr code generation script:
\begin{lstlisting}[language=C++]

#define try_expand(value, head, ...)            \
if constexpr (aggregate_size<decltype(value)>() \
              == tokcount(+1, __VA_ARGS__)) {   \
  auto [head __VA_OPT__(,) __VA_ARGS__] = value;\
  return tie(head __VA_OPT__(,) __VA_ARGS__);   \
}                                               \
__VA_OPT__(__THIS_MACRO__(value, __VA_ARGS__))

constexpr auto tie_aggregate(auto&& aggregate) {
  /* expands to similar functional
     as boost::pfr for up to 3 members */
  try_expand(aggregate, _3, _2, _1);
}

\end{lstlisting}

\newpage

\section{Proposed wording}

\href{https://eel.is/c++draft/cpp#replace.general-6}{[cpp.replace.general]}

\begin{quote}
The identifiers \textcolor{green}{ \lstinline{__THIS_MACRO__},}
\lstinline{__VA_ARGS__} and \lstinline{__VA_OPT__}
shall occur only in the replacement-list
of a function-like macro that uses
the ellipsis notation in the parameters.
\textcolor{green}{Token \lstinline{__THIS_MACRO__}
behaves as if it is replaced by name of macro
in which definition it appears, except name
of such macro always replaced during [cpp.rescan]}
\end{quote}

\bigskip

\href{https://eel.is/c++draft/cpp#rescan-3}{[cpp.rescan]}

\begin{quote}
If the name of the macro being replaced is found during this scan of the replacement list (not including the rest of the source file's preprocessing tokens)
\textcolor{green}{and its definition does not contain \lstinline{__THIS_MACRO__} token, }
it is not replaced. Furthermore, if any nested replacements encounter the name of the macro being replaced, it is not replaced. These nonreplaced macro name preprocessing tokens are no longer available for further replacement even if they are later (re)examined in contexts in which that macro name preprocessing token would otherwise have been replaced.
\end{quote}

\section{Implementation experience}

The implementation in clang and examples,
could be found \href{https://github.com/llvm/llvm-project/pull/65851}{here[link]}

\end{document}
